\documentclass[a4paper]{article}
\usepackage[parfill]{parskip}
\usepackage{graphicx}
\usepackage{caption}
\usepackage{subcaption}
\usepackage[T1]{fontenc}
\usepackage{enumerate}
\usepackage{mathtools}
\usepackage{fancyref}
\usepackage{amsmath}
\usepackage[utf8]{inputenc}
\usepackage[english]{babel}
\usepackage[usenames, dvipsnames]{color}
\usepackage{rotating}
\usepackage{comment}

\begin{document}

\section{Implied Volatility}



The Black-Scholes option pricing formula depends on \textit{S}, \textit{E}, \textit{r}, $\tau$, and $\sigma^2$. The asset volatility, $\sigma$, is the only parameter that cannot be observed directly. In practice, analysts use the observed market price, and  knowing \textit{S}, \textit{E}, \textit{r} and $\tau$, they back out the volatility when using the Black-Scholes model. The volatility obtained using this methodology is referred to as the \textit{implied volatility}. The implied volatility can vary, contradicting the assumption of constant volatility. This phenomenon is known as the volatility smile.

The smile first appeared in options markets in 1987. Pre 1987 financial crash, implied volatility followed the Black-Scholes model attributing a single lognormal volatility to an underlying stock at all times and all strike prices \cite{Smile}. Therefore, predicting a flat implied volatility surface. Post 1987, index options' volatility surfaces have become skewed. This has since spread to stock options, currency option, and interest rate options.

A common explanation for the smile in equity options concerns leverage. As the equity of a company declines in value, the leverage increases. In doing so, it's volatility increases due to the equity becoming more risky. Conversely, as the equity of a company increases in value, leverage decreases. Resulting in the equity becoming less risky and its volatility decreases.

\subsection{Black-Scholes Implied Volatility} \label{sec:BS_Imp_Vol_New}

From the market data we are able to assigned values to \textit{E}, \textit{r}, \textit{T} and \textit{S}. Thus, the volatility is the only unknown parameter and therefore the option value can be treated as a function of $\sigma$, denoted C($\sigma$), where the quoted value of the call option is denoted C($\sigma^*$). The aim is to find the implied volatility, $\sigma$, that solves C($\sigma$)=C($\sigma^*$).

We shall use Newton's method to solve this nonlinear equation. Newton's method takes the form:

\begin{equation*}
	\sigma_{(n+1)}=\sigma_n-\frac{F(\sigma_n)}{F'(\sigma_n)},
\end{equation*}

where F($\sigma$)=C($\sigma$)-C($\sigma^*$)=0, and F'($\sigma$)=$\partial$C/$\partial \sigma$. This process is iterated until a certain tolerance in met.

For the initial guess of the implied volatility, we use

\begin{equation*}
	\hat{\sigma}=\sqrt{2\bigg|\frac{log(S/E)+r(T-t)}{T-t}\bigg|}.
\end{equation*} 

We use this value as our initial guess as this value maximises the differential with respect to $\sigma$ ($\partial C/$$\partial \sigma$) over $[0,\infty)$.

 Additionally, it ensures that the error in $\sigma_{n+1}$ is smaller than, but has the same sign (positive) as, the error in $\sigma_n$ \cite{Higham_2004}. Furthermore, the error decreases monotonically as \textit{n} increases.


\subsection{Derivation of Initial Implied Volatility Guess}

The first and second differentials of the cumulative normal density function are the following:

\[ N'(x) = \frac{1}{\sqrt{2\pi}}e^{-\frac{1}{2}x^2} \quad \& \quad N''(X) = \frac{-x}{\sqrt{2\pi}}e^{-\frac{1}{2}x^2}. \]

From Index (\ref{sec:Greeks}) we have the following equation for the differential of a European call option with respect to $\sigma$:

\[ \frac{\partial C}{\partial \sigma} = \sqrt{T-t}SN'(d_1). \]	

To find the maximum of $\frac{\partial C}{\partial \sigma}$ we have to differentiate again with respect to $\sigma$:

\[ \frac{\partial^2 C}{\partial \sigma^2} = \sqrt{T-t}SN''(d_1)\frac{\partial d_1}{\partial \sigma}. \]	

Additionally, we have the following differential of $d_1$ with respect to $\sigma$:

\begin{align*}
	\frac{\partial d_1}{\partial \sigma} & = \frac{-log(S/E)}{\sigma^2 \sqrt{T-t}} + \frac{T-t}{\sqrt{T-t}}\big(-\frac{r}{\sigma^2} + \frac{1}{2} \big) \\
	& = -\left[ \frac{log(S/E) + r(T-t) - \frac{\sigma^2}{2}(t-t)}{\sigma^2\sqrt{T-t}} \right] \\
	& = - \left[ \frac{log(S/E) + (r-\frac{\sigma^2}{2})(T-t)}{\sigma^2\sqrt{T-t}} \right] \\
	& = -\frac{d_2}{\sigma}. \\
\end{align*}
Giving:
\[ \frac{\partial^2 C}{\partial \sigma^2} = \frac{S\sqrt{T-t}}{\sqrt{2\pi}}e^{-\frac{1}{2}}d_1^2\frac{d_1d_2}{\sigma}. \]
$\frac{\partial^2 C}{\partial \sigma^2}=0$ when $d_1d_2=0$. Recall $d_1$ and $d_2$:
\begin{align*}
	d_1 & = \frac{log(\frac{S}{E}) + (r + \frac{1}{2} \sigma^2)(T-t)}{\sigma \sqrt{T-t}}, \\
	d_2 & = \frac{log(\frac{S}{E}) + (r - \frac{1}{2} \sigma^2)(T-t)}{\sigma \sqrt{T-t}}.
\end{align*}
$d_1=0$ when:
\[ \sigma = \sqrt{-2 \left[ \frac{log(S/E) + r(T-t)}{T-t}\right]}. \] 
$d_2=0$ when:
\[ \sigma = \sqrt{2 \left[ \frac{log(S/E) + r(T-t)}{T-t}\right]}. \]
Therefore, $d_1d_2=0$ when:
\[ \sigma = \sqrt{2 \left| \frac{log(S/E) + r(T-t)}{T-t} \right| }. \]



\newpage



\begin{thebibliography}{99}

\bibitem{Smile} \textbf{Emanuel Derman (2003).} "Laughter In The Dark - The Problem Of The Volatility Smile."

\bibitem{Higham_2004}\textbf{Higham, D. J. (2004).} "An Introduction To Financial Option Valuation: Mathematics, Stochastics And Computation.", p.131-140.


\end{thebibliography}
























	
\end{document}
